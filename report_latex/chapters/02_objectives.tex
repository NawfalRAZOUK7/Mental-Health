\chapter{Objectifs et questions de recherche}
\section{Objectifs}
\begin{itemize}
    \item O1 : Décrire les profils du suicide à l'échelle mondiale et régionale.
    \item O2 : Comparer les charges liées à la dépression, aux addictions et à l'automutilation.
    \item O3 : Explorer les relations entre indicateurs et construire des modèles prédictifs de base.
    \item O4 : Démontrer des techniques de BI et de data mining sur des données de santé publique.
    \item O5 : Garantir la reproductibilité avec des pipelines et des versions clairement séparées.
\end{itemize}

\section{Questions de recherche}
\begin{itemize}
    \item Q1 : Quelles sont les différences géographiques majeures des taux de suicide et comment se distribuent-elles par région ?
    \item Q2 : Comment se positionnent les pays sur les indicateurs de dépression (DALYs), d'addictions et d'automutilation ?
    \item Q3 : Observe-t-on des relations cohérentes entre les indicateurs (suicide vs dépression, addictions, automutilation) ?
    \item Q4 : Un modèle prédictif simple peut-il estimer un risque relatif à partir des indicateurs disponibles ?
    \item Q5 : Quelles techniques avancées (clustering, détection d'anomalies) apportent une valeur ajoutée en analyse exploratoire ?
\end{itemize}
