% Front matter. Use \title, \author, and \date in main.tex.
% If you want an unnumbered chapter here, use \chapter*{Front Matter} in main.tex.
\section*{Résumé}
Ce rapport présente une plateforme d'analyse et de visualisation autour de la santé mentale, avec un focus sur le suicide, la dépression, les addictions et l'automutilation. Les données proviennent de l'OMS (suicide 2021) et de l'IHME GBD (2023) et sont structurées en trois versions : v0 (visualisations statiques), v1 (tableau de bord principal sur données réelles) et v2/v3 (analyses avancées et estimation de risque). Le pipeline inclut un inventaire des sources, un mapping ISO3, un nettoyage par fichier, puis des fusions pour construire une table ML cohérente. Le tableau de bord v1 fournit des cartes, des classements et des comparaisons par sexe et âge, ainsi que des pages de contexte (tendances toutes causes, grandes catégories, probabilité de décès). La partie analytique propose des corrélations, une démonstration ML (Ridge, RandomForest) et des techniques de data mining dans v2. Le rapport insiste sur la reproductibilité, les limites de fusion inter-années et les risques d'interprétation. L'objectif est de fournir une lecture claire, professionnelle et exploitable des indicateurs de santé mentale à l'échelle mondiale.

\section*{Abstract}
This report presents a mental health analytics and visualization platform focused on suicide, depression, addictions, and self-harm. Data sources combine WHO suicide statistics (2021) and IHME GBD indicators (2023), organized into three project versions: v0 (static visuals), v1 (main dashboard on real data), and v2/v3 (advanced analytics and risk estimation). The pipeline covers inventorying, ISO3 mapping, per-file cleaning, and merges to build a consistent ML table. The v1 dashboard delivers maps, rankings, and sex/age comparisons, plus contextual pages on all-cause trends, big categories, and probability of death. The analytics layer adds correlations, an ML demo (Ridge, RandomForest), and data mining techniques in v2. The report highlights reproducibility, cross-year merge limits, and interpretation risks. The goal is to provide a clear, professional, and actionable view of global mental health indicators.

\section*{Mots-clés / Keywords}
\begin{tabular}{p{0.46\textwidth} p{0.46\textwidth}}
\textbf{Français} & \textbf{English} \\
santé mentale & mental health \\
suicide & suicide \\
dépression & depression \\
addictions & addictions \\
automutilation & self-harm \\
OMS & WHO \\
IHME GBD & IHME GBD \\
visualisation & visualization \\
\end{tabular}
