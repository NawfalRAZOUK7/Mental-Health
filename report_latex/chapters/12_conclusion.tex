\chapter{Conclusion et perspectives}
\section{Constats principaux}
Ce projet met en évidence la forte hétérogénéité des taux de suicide et des indicateurs associés selon les pays, les régions, le sexe et l'âge. Les tableaux de bord v1 offrent une lecture claire des tendances et des comparaisons internationales, tandis que les couches analytiques (v2/v3) illustrent la valeur d'approches avancées pour la segmentation, la détection d'anomalies et l'estimation de risque.

\section{Perspectives}
Plusieurs extensions sont envisageables :
\begin{itemize}
    \item Intégrer des années supplémentaires (séries longues) pour consolider l'analyse temporelle.
    \item Enrichir les facteurs explicatifs (variables socio-économiques, services de santé, indicateurs d'accessibilité).
    \item Valider les modèles sur des données externes ou des études de cas nationales.
    \item Ajouter des mécanismes de gouvernance des données (audit automatique, versioning avancé).
\end{itemize}
