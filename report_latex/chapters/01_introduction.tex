% NOTE : si vous utilisez la classe article, remplacez \chapter par \section.
\chapter{Introduction}
\section{Motivation}
La santé mentale est un enjeu mondial majeur, à la fois médical, social et économique. Les troubles dépressifs, les addictions et l'automutilation affectent des millions de personnes et fragilisent les systèmes de santé. En parallèle, le suicide reste une cause importante de mortalité évitable. Disposer d'une lecture claire, comparative et accessible de ces indicateurs est essentiel pour orienter la prévention et les politiques publiques.

\section{Pourquoi le suicide et les troubles mentaux comptent}
Le suicide représente un indicateur critique de détresse psychologique et sociale. Les troubles mentaux et les troubles liés aux substances constituent des facteurs de risque majeurs et contribuent aussi à la charge de morbidité (DALYs). Comprendre comment ces dimensions évoluent dans l'espace (pays, régions) et selon les caractéristiques démographiques (sexe, âge) permet d'identifier des zones à risque, des populations vulnérables et des pistes d'action.

\section{Périmètre et public cible}
Ce rapport présente une analyse multi-couches combinée de données OMS et IHME GBD, avec trois versions du projet : v0 (visualisations statiques), v1 (tableaux de bord sur données réelles) et v2/v3 (analyses avancées et outils d'aide à la décision). Le public cible est double : d'une part les enseignants et évaluateurs, d'autre part les utilisateurs non techniques qui ont besoin d'une lecture intuitive des tendances et des relations entre indicateurs.
